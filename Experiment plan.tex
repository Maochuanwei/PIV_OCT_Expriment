\documentclass[12pt]{article}
\usepackage{ctex} % 支持中文
\usepackage{geometry}
\usepackage{amsmath}
\usepackage{graphicx}
\usepackage{booktabs}
\usepackage{hyperref}
\usepackage{float} % 用于控制表格位置
\usepackage{longtable}


\geometry{a4paper, margin=0.4in,bottom=0.5in}

\title{OCT-PIV流速测量实验}
\author{毛川伟}
\date{\today}

\begin{document}
\maketitle

\section{研究背景}

PIV,全称为微粒子图像测速。基本原理是通过一个相机连续拍摄在溶液当中的示踪粒子。通过对比连续拍摄到的图片,计算前一张图片到后一张图片的相关性。找到相关位移。然后通过标定,得到实际和像素之间的关系
通过这个关系来计算流速。下面是PIV的基本原理图

\begin{figure}[h]
    \centering
    \includegraphics[width=0.8\textwidth]{Images/PIV_principles.png}
    \caption{粒子图像测速技术工作原理示意图}
    
    
\end{figure}

PIV的计算公式如下,


    \begin{equation}
        R(x,y)=\sum_{i=-L}^L\sum_{j=-L}^LI(i,j)I'(i+x,j+y)
    \end{equation}

    其中$I(.)$和$I'(.)$分别为前后帧图像提取的窗口像素灰度值。第一幅图中的查询窗口称为模板,利用模板在第二幅图像的搜索范围中进行不断的遍历和匹配,就
    可以得到相关平面

    \begin{figure}
        \centering
        \includegraphics[width=0.8\textwidth]{Images/互相关算法示意图.png}
        \caption{互相关 PIV 算法的计算原理示意图}
        
        
    \end{figure}

    \begin{figure}
        \centering
        \includegraphics[width=0.8\textwidth]{Images/互相关示意图.png}
        \caption{互相关算法计算的互相关平面}
        
    \end{figure}
    

\section{实验目的}
\begin{itemize}
    \item 测试PIV测速的可行性
    \item 验证PIV

\end{itemize}

\section{实验内容}
\begin{itemize}

    \item 制作玻璃管流道
    \item 使用聚苯乙烯微球作为示踪粒子,配置聚苯乙烯微球溶液
    \item 设置不同的灌流流速,使用OCT设备采集数据
    \item 使用PIVlab工具包计算流场和速度
    \item 将得到的流场和预先设置值进行对比,计算均值和方差,分析该方法得到的结果
\end{itemize}



\section{实验材料和设备}
\subsection{材料清单}



{
    \footnotesize
    \begin{longtable}{@{} p{5cm} p{10cm} @{}}
        \toprule
        \textbf{材料} & \textbf{备注} \\ 
        \midrule
        
        
        聚苯乙烯微球        & 用来作为示踪粒子\\
        玻璃管两根          & 作为微流道,一根备用\\
        橡胶管两条              & 用来连接注射器,微流道,和灌流器 \\
        灌流器              & 用来设置灌流流速 \\
        琼脂粉              & 和水混合加热到沸腾,等到30-40°冷却就会凝固\\
        一个离心管的瓶盖    & 用于放置玻璃管\\
        微波炉              & 用于加热琼脂 \\
        天平                 & 在使用之前需要在放上称量纸的情况下调零 \\ 
        移液枪               & 在配置材料实验室里 \\ 
        离心管(2个)        & 用于配置聚苯乙烯微球溶液 \\ 
        称量纸               & 方便倒出材料   \\
        5ml 注射器(4个)    & 存放聚苯乙烯微球溶液 \\ 
        灌流器               & 位于三楼实验室抽屉 \\ 
        空心橡胶管(2条)    & 长度不能够太短 \\ 
        3D 打印注射器喷头    & 红色,套在注射器上用于连接橡胶管 \\ 
        纯净纸巾             & 用来擦拭漏出的溶液 \\ 
        
        一把剪刀            & 用来剪断橡胶管 \\
        
        \bottomrule
        \caption{实验材料列表} \\
    \end{longtable}
}




\section{实验流程}

\subsection{准备材料}
\begin{itemize}
  
    \item 提前带上U盘或者移动硬盘
    \item 提前预约实验室
    \item 提前打开实验室的新风系统(中间控制面板),以及空调系统(在门左右边)
    \item 带上聚苯乙烯微球材料
\end{itemize}
\subsection{配材料}
\begin{enumerate}
    \item 琼脂溶液的配置
    \begin{itemize}
        \item 百分之5的琼脂溶液,取10g琼脂粉,灌水到200ml
        \item 加入一点牛奶(不能够加太多,会降低透光度)
        \item 使用微波炉加热琼脂到沸腾
       

    \end{itemize}
    \item 制作带缺口瓶盖
    \begin{itemize}
        \item 取下一个离心管的瓶盖
        \item 使用小刀刻出缺口
    \end{itemize}

    \item 配置聚苯乙烯微球溶液
    \begin{itemize}
        \item 配置百分之10质量的微球溶液
        \item 取5g聚苯乙烯微球,使用移液枪添加45ml取离子水
        \item 需要搅拌均匀
    \end{itemize}
\end{enumerate}

\subsection{开始实验}
\begin{enumerate}
    \item 材料的装配
    \begin{itemize}
   
    \item 使用5ml的注射器吸入聚苯乙烯微球溶液
    \item 将材料都放到同一个实验架子上,并做好区分,这样不用到处找材料
    
    \end{itemize}

    \item 制作直流通道
    \begin{itemize}
        \item 从微波炉当中取出溶液,倒入瓶盖当中并等待凝固(注意琼脂不要太厚,应该稍微覆盖即可,否则会降低透光率)
        \item 连接金属管和橡胶管
    \end{itemize}

    \item 搭建灌流平台
    \begin{itemize}
        
        \item 将直流通道和灌流设备连接起来
        \item 往流道里面注入溶液
        \item 设置不同的灌流流速
        
       这个位置用来放置流速表格,流速尽量设置得比较低一点,可以用来验证方法的可行性
        


    
    \end{itemize}

    \item 开始采集数据
    \begin{itemize}
        \item 等到流速稳定之后开始采集数据
        \item 找到像,我们应该找到正像,通过显微镜镜头找到正像(从摄像头看,显微镜达到对扫描物体的聚焦比较清楚),然后将物体移动到强度最高的位置(注意,会呈现两个像,一个正像一个反像,这是由于光程差存在正反
        (注意:对于具有强反射率的玻璃表面,我们的成像可能会存在两条横杠,这个玻璃的反光,可能会影响最终的成像)
        \item 调整像素个数,让成像更加清晰,或者通过固定像素实际大小和视场来自动得到像素值(会导致文件太大,所以需要提前准备磁盘,读写速度较大的磁盘)
        \item 开始采集3D散斑数据,建立excle表格记录每个灌流量的采集数据(包括流速设置,OCT参数,显微镜照片)
        \item 每个灌流量,需要重复采集4次(为了后面建立方差和均值模型)
        \item 实验采集数据完毕
    \end{itemize}

    \item 实验结束准备
    \begin{itemize}
        \item 将不要的东西都扔到垃圾桶当中
        \item 将灌流器收起来放到抽屉当中
        \item 关闭实验室的新风系统,关闭实验室的空调(新风系统显示连锁关闭,空调在左右门)
        \item 将OCT设备关机(注意,先关闭计算机,才关闭OCT设备)
        \item  将垃圾到到6楼的垃圾筒当中去
    \end{itemize}

    \section{数据处理部分}
    \subsection{Matlab处理部分}
    \begin{enumerate}
        \item 将从公司得到的数据放到自己电脑上
        \item 先运行制作掩膜的代码,会自动保存到电脑中
        \item 然后运行制作CC值的代码,只要指定文件夹,其他的都是自动运行
        \item 然后是CC值处理代码,会保存未滤波后滤波后的图片(可以添加功能,将这个滤波后的CC值也保存一下)
        \item 处理完毕
    \end{enumerate}
    \subsection{Amira处理部分}
    \begin{enumerate}
        \item 将图片导入到Amira当中,并打上标签,备注是什么灌流量的
        \item 选择伪彩图映射函数Physical,并不断调整对比度,直到达到一个看起来像一个流场的环境,并能够区分不同组别的灌流量
        \item 使用不同窗口进行展示,然后使用软件内截图功能以及文本标记功能对图像打上标签(使用外部截图的分辨率太低,存在对齐问题,不推荐)
    \end{enumerate}
    

\end{enumerate}





\section{可能的意外与应对措施}
\begin{itemize}

    \item 配材料室没开门(等待 可能有人在休息)
    \item 师兄没有时间做实验(等待下一次做实验)
    \item 中间遇到的会议(看情况,如果会议紧急的话先去开会议,否则可以请假)
\end{itemize}

\section{参考文献}

\begin{itemize}
    \item 基于 CCd-OCTA 的三维流场成像技术及其对3D打印类血管网的研究 
    \item Vista
    \item Complex differential variance algorithm for optical coherence tomography angiography
    \item Improved motion contrast and processing efficiency in OCT angiography using complex-correlation algorithm
\end{itemize}

\end{document}